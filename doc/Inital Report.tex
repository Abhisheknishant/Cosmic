\documentclass[conference]{IEEEtran}
\IEEEoverridecommandlockouts
% The preceding line is only needed to identify funding in the first footnote. If that is unneeded, please comment it out.
\usepackage{cite}
\usepackage{amsmath,amssymb,amsfonts}
\usepackage{algorithmic}
\usepackage{graphicx}
\usepackage{textcomp}
\usepackage{xcolor}
\def\BibTeX{{\rm B\kern-.05em{\sc i\kern-.025em b}\kern-.08em
    T\kern-.1667em\lower.7ex\hbox{E}\kern-.125emX}}
\begin{document}

\title{
 Cosmic\\A Software Simulated  Computing Platform}

\author{\IEEEauthorblockN{Clay Buxton}
\IEEEauthorblockA{\textit{Computer Engineering, Computer Science} \\
\textit{Elizabethtown College}\\
Elizabethtown, PA \\
buxtonc@etown.edu}
\and
\IEEEauthorblockN{Kevin Carman}
\IEEEauthorblockA{\textit{Computer Engineering, Computer Science} \\
\textit{Elizabethtown College}\\
Elizabethtown, PA \\
carmank@etown.edu}

}

\maketitle

\begin{abstract}
Cosmic is a software simulated, from-scratch, 8-bit microcomputer designed to resemble retro computers. Cosmic consists of many parts including a simulated CPU, underlaying system, inputs and outputs, and graphics. In addition to the simulated machine, we plan to write software that targets the device. The goal of Cosmic is to develop a system that is equivalent to an Apple II, TRS-80, and similar machines of the early 1980s
\end{abstract}

\begin{IEEEkeywords}
retrocomputing, simulation, emulation, somethingelse
\end{IEEEkeywords}

\section{Introduction}
The cosmic project can be broken down into X main parts
\begin{itemize}
	\item \textbf{Cosmic Processor} - The simulated CPU used to drive the rest of the system. The CosmicCPU is comparable to the Zilog Z80 and the MOS 6502.
	\item \textbf{Cosmic System} - This is everything that is found on the mainboard of a similar  physical machine, things like memory, input \& output, graphics, and audio. This also will fit in with appropriate machines like the Apple II.
	\item \textbf{Cosmic Software}  - This portion is software written that is not directly part of cosmic. This includes an Assembler for the Cosmic Instruction set and software written that is targeted to run on a Cosmic system.
	\item \textbf{Auxillary Software} - This porition is the software written to run on a modern computer. This will be the actual simulation environment and an assembler for Cosmic
\end{itemize}

Once finished the simulated system will operate similarly to a computer of the early 1980's, primarily the Apple II. 

\section{Background}

The largest source of background information is in the machines that Cosmic is made to fit in with. As previously mentioned the two primary reference devices are the Apple II and the TRS-80 containing the MOS 6502 and the Zilog Z80 respectively. 




\end{document}