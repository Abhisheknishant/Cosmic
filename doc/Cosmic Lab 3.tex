%%%%%%%%%%%%%%%%%%%%%%%%%%%%%%%%%%%%%%%%%
% fphw Assignment
% LaTeX Template
% Version 1.0 (27/04/2019)
%
% This template originates from:
% https://www.LaTeXTemplates.com
%
% Authors:
% Class by Felipe Portales-Oliva (f.portales.oliva@gmail.com) with template 
% content and modifications by Vel (vel@LaTeXTemplates.com)
%
% Template (this file) License:
% CC BY-NC-SA 3.0 (http://creativecommons.org/licenses/by-nc-sa/3.0/)
%
%%%%%%%%%%%%%%%%%%%%%%%%%%%%%%%%%%%%%%%%%

%----------------------------------------------------------------------------------------
%	PACKAGES AND OTHER DOCUMENT CONFIGURATIONS
%----------------------------------------------------------------------------------------

\documentclass[
	12pt, % Default font size, values between 10pt-12pt are allowed
	%letterpaper, % Uncomment for US letter paper size
	%spanish, % Uncomment for Spanish
]{fphw}

% Template-specific packages
\usepackage[utf8]{inputenc} % Required for inputting international characters
\usepackage[T1]{fontenc} % Output font encoding for international characters
\usepackage{mathpazo} % Use the Palatino font

\usepackage{graphicx} % Required for including images

\usepackage{booktabs} % Required for better horizontal rules in tables

\usepackage{listings} % Required for insertion of code

\usepackage{enumerate} % To modify the enumerate environment

\usepackage{hyperref}
\hypersetup{
    colorlinks=true,
    linkcolor=blue,
    filecolor=magenta,      
    urlcolor=cyan,
}

%----------------------------------------------------------------------------------------
%	ASSIGNMENT INFORMATION
%----------------------------------------------------------------------------------------

\title{Assembly Lab \#3: Raspberry Pi Interfacing} % Assignment title


\date{March 28th, 2025} % Due date

\institute{Elizabethtown College \\ Department of Computer Science} % Institute or school name

\class{Digital Design II} % Course or class name

\professor{Professor X} % Professor or teacher in charge of the assignment

%----------------------------------------------------------------------------------------

\begin{document}

\maketitle % Output the assignment title, created automatically using the information in the custom commands above

%----------------------------------------------------------------------------------------
%	ASSIGNMENT CONTENT
%----------------------------------------------------------------------------------------

\section*{Lab Objective}

\begin{problem}
	Interface between Cosmic and the physical world 
\end{problem}

%------------------------------------------------

\subsection*{Prelab}\\

\begin{itemize}
  \item Set up the Cosmic environment and assembler on the Raspberry Pi
  \item Read the documentation on the interfacing
\end{itemize}

%----------------------------------------------------------------------------------------

\section*{During Lab}

\begin{problem}
	You will write a simple program that will take input from a button hooked up to the Raspberry Pi and light up a pixel on the screen when it is active.
\end{problem}

%------------------------------------------------

\begin{itemize}
  \item Poll memory for input from the button
  \item Once input is detected write a pixel to the screen
  \item If input is no longer detected remove the pixel
\end{itemize}

%----------------------------------------------------------------------------------------

\section*{Grading}

30\% - Program Assembles\\
30\% - Shows pixel on input high
30\% - Removes pixel on input low\\
10\% - Comments Added\\



%------------------------------------------------

\section*{Helpful Links}
\href{https://github.com/clbx/Cosmic/tree/master/doc}{Cosmic Documentation}\\

%----------------------------------------------------------------------------------------

\end{document}
